\chapter{Introduction}

CosmicWatch: The Desktop Muon Detectors \cite{axani2019physics}, are a self-contained, low-cost, and easy-to-build particle detector for students, scientists, and cosmic-ray enthusiasts. It aims to make particle detection interactive and available to anyone interested in learning about the electronics and physics involved in this area of expertise. With this in mind, the detector design prioritizes the user experience across the board, from its construction to data acquisition and processing. It uses a silicon photomultiplier (SiPM) to collect light emitted by a plastic scintillator after a charged particle, like a cosmic-ray muon, deposits some of its energy in it. This project aims to further expand the capabilities of CosmicWatch by exploiting the already existing electronics and implementing the necessary features to transform the detector into a portable gamma-ray spectrometer

Using a Cerium doped Lutetium-based scintillation crystal (LYSO), we have achieved an energy resolution of  $4.86 \sqrt{E~[\unit{\MeV}]}$ while testing in a Rohde\&Schwarz RTO6 oscilloscope to sample the data.

The human body is known to have many limitations, our senses are often not the best tools to delve into the intricacies of nature. For many years scientists have taken advantage of the sensitivity of materials to further expand our capabilities to explore the world around us, bringing to our reach worlds once invisible. Scintillating crystals for example have allowed us to develop a type of detector able to distinguish the energy deposition in it, making elusive particles trackable, no longer letting them escape our curiosity. The wonders of these types of detectors are sadly not easily available to everyone, scintillating and solid-state detectors are often out of the economic capacities of most. CosmicWatch Desktop Muon Detectors are therefore an extremely powerful tool to bring particle detection closer to the public, students, and young scientists like myself. This work thus aims to further explore and expand the capabilities of CosmicWatch, hoping to one day provide a self-contained, low-cost, and easy-to-build particle detector suited for gamma-ray spectroscopy.