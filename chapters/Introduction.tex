\chapter{Introduction}

CosmicWatch: The Desktop Muon Detectors \cite{axani2019physics}, is a self-contained, low-cost, and easy-to-build particle detector for students, scientists, and cosmic-ray enthusiasts. It aims to make particle detection interactive and available to anyone interested in learning about the electronics and physics involved in this area of expertise. With this in mind, the detector design prioritizes the user experience across the board, from its construction to data acquisition and processing. It uses a silicon photomultiplier (SiPM) to collect light emitted by a plastic scintillator after a charged particle, like a cosmic-ray muon, deposits some of its energy in it. This project aims to further expand the capabilities of CosmicWatch by exploiting the already existing electronics and implementing the necessary features to transform the detector into a portable gamma-ray spectrometer

So far, previous iterations of CosmicWatch have worked as a Geiger counter, providing information about the number of particles detected, but ignoring energy deposition due to the poor resolution of the scintillating material used, generally plastic scintillators such as BC-404. The current speed of the electronics used is also a limiting factor, creating deadtime and therefore decreasing the sample rate. By switching to a Cerium-doped Lutetium-based scintillation crystal (LYSO), testing new electronics, and implementing better programming paradigms, this work thus aims to further explore and expand the capabilities of CosmicWatch, hoping to one day provide a self-contained, low-cost, and easy-to-build particle detector suited for gamma-ray spectroscopy.