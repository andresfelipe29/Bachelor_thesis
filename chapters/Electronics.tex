\chapter{Electronics}
\label{chap:Electronics}

CosmicWatches have to be mainly low-cost and easy to build. In order to achieve this, the components selected for the construction have been carefully curated to make sure these restrictions were met. Some of these components, however, have been found to introduce odd behaviors in the signal processing, this represents difficulties while testing the detector, some examples are the lack of linearity \ref{sec:Non-linearity}, and fluctuations in amplification and peak-detected values of seemingly equal input signals. The full KiCad project can be found in the GitHub repository: \href{https://github.com/anvargasl/CosmicWatch-gamma-spectroscopy-PCB}{CosmicWatch-gamma-spectroscopy-PCB}. The component labels shown in this chapter are also the ones used in the KiCad project, meaning that they represent the components that would have to be placed on the PCB in order to recreate the example schematics.

\begin{figure}[H]
    \centering
    \includegraphics[width=0.73\textwidth]{Electronics/CW-signals.pdf}
    \caption{Signal processing done by the CosmicWatch electronics.}
    \label{fig:signal_processing}
\end{figure}

The processing of a pulse coming out of the SiPM has to go through two main stages, amplification and peak detection, Fig. \ref{fig:signal_processing} showcases these stages. The brightest SiPM pulses seen so far do not exceed 200 mV, which covers a very small portion of the ADC range on the RP Pico (0-3.3 V \cite[p.~18]{datasheet2024RpPico}). Amplification of the signal therefore allows for better resolution. As Fig. \ref{fig:signal_processing} illustrates, a SiPM pulse fully decays approximately after 50 \unit{\micro\s}, this means that the microprocessor will most likely not sample the actual peak value, it could even read the voltage long after the pulse has decayed depending on the response time. In order to give the microprocessor enough time to get an accurate reading of the amplitude, the amplified pulse is stretched in time (\textcolor{orange}{orange signal}), making sure that the sampled voltage is near the peak value even if the reading is delayed.

\section{Amplifier}

\begin{figure}[H]
    \centering
    \begin{circuitikz}[scale=0.7]
        \draw
        (0,0) node[op amp, noinv input up, font=\small] (opamp) {$U6$}
        (opamp.up) --++(0,0.5) node[vcc]{5\,\textnormal{V}}
        (opamp.down) --++(0,-0.5) node[vee]{-5\,\textnormal{V}}
        (opamp.+) node[left]{$V_{in}$}
        (opamp.-) node[left]{} to[short] ++(0, -3) coordinate(R4_R)
        to[R, l_=$R_4$] ++(-3, 0) node[ground]{}
        (R4_R) to[R, l_=$R_6$] ++(2.7, 0)
        %to[short] ++(0.1, 0) -|(opamp.out) to[short] ++(1, 0) node[ocirc, label={[yshift=0.3]$V_{out}$}]{}
        to[short] ++(0.1, 0) -|(opamp.out) to[short,-*] ++(1,0) node[above]{$V_{out}$};
        %to[short] ++(1, 0) node[ocirc, label=above:$V_{out}$]{};

        %Nodes
        %\node[shift={(0,+1.5)}] at (opamp) {U6 HLM6658};
    \end{circuitikz}
    \caption{Amplifier circuit schematic. An LT1807 op-amp is used for this and the peak detection stage.}
    \label{circ:amplifier}
\end{figure}

An op-amp on its own amplifies the voltage difference between the non-inverting (pin $+$) and inverting (pin $-$) inputs by its internal gain $A_{int}$, having then $V_{out}=A_{int}(V_+ - V_-)$. In this case, however, we are interested in controlling the gain of the circuit and therefore the amplification. In order to achieve this we introduce a feedback loop in the op-amp through $R4$ and $R6$, which controls how much of the output voltage is fed back into the op-amp. The theoretical amplification is therefore given by $V_{out}=(1+R6/R4)V_{in}$. A simple schematic showcasing the component arrangement is shown in Fig. \ref{circ:amplifier}.

\section{Peak Detector}

Since the LYSO crystal is so fast (36 ns of decay time \cite{Luxium_LYSO}) the ADC sample rate and response time of the Pico both play an important role in the number of events that the detector will accurately acquire. It is therefore necessary to hold the voltage of the amplified pulse in order to increase the chances of reading the actual value of the incoming signal. This is the task of the peak detector, to widen the time window in which we can sample the ADC and get a correct reading.

The idea behind the peak detector is to store charge in a capacitor ($C_{25}$) through a diode ($D_3$), retaining the highest voltage the input signal has reached. A diode is placed before the capacitor so that once the signal's voltage goes below the peak voltage, the diode will be reverse biased, therefore preventing current from flowing while maintaining the voltage on the capacitor.

In order to measure the voltage in the capacitor, a discharging resistor has to be added ($R_{15}/R_{19}$). The time it takes the capacitor to discharge is given by $t=RC$. Although for example, in the case of CosmicWatch-V2's peak detector, the values of $R_{14}$ and $R_{24}$ also play a role in the discharging time, which has proved not to be as trivial as calculating the equivalent resistance $R_t$ of all three and simply take $t=R_tC$.

The schematic and PCB shown in the repository \href{https://github.com/anvargasl/CosmicWatch-gamma-spectroscopy-PCB}{CosmicWatch-gamma-spectroscopy-PCB}, include the connections and footprints necessary to place the components that make the designs illustrated in Sections \labelcref{sec:basic,sec:pd_V2,sec:basic_buffer,sec:nuclear_phoenix}. Different results were found while testing these peak detector setups, none of them, however, were capable of holding the true value of the amplified pulse and did not respond linearly. In our case, the nonlinearity meant that a 2 mV increase in SiPM pulses bellow 50 mV produced an appreciable change in the peak detected signal, while for bigger pulses (above 50 mV) a significant change occurred only after a 10 mV increase (see Figure \ref{fig:nonlinearity}). These problems need to be addressed in order produce good spectra since this greatly affects energy resolution.

\subsection{Basic Peak Detector}\label{sec:basic}

\begin{figure}[H]
    \centering
    \begin{circuitikz}[scale=0.7]
        %\draw [help lines] (-4,0) grid (5,-4);
        \draw (0,0) node[op amp, noinv input up, font=\small] (opamp) {$U6$}
        (opamp.up) --++(0,0.5) node[vcc]{5\,\textnormal{V}}
        (opamp.down) --++(0,-0.5) node[vee]{-5\,\textnormal{V}}
        (opamp.+) node[left]{$V_{in}$}
        (opamp.out) node[]{} to[sD, l_=$D_3$] (6, 0) coordinate (D3_r)
        (opamp.-) node[left]{} to[short] ++(0, -2.5) coordinate (R24_l)
        (R24_l) to[R, l^=$R_{24}$, resistors/scale=0.8] (D3_r |- , |- R24_l) coordinate (R24_r)
        (D3_r) -- (R24_r)
        (D3_r) -- ++(1,0) coordinate (C25_u)
        (C25_u) to[C, l=$C_{25}$] ++(0,-2) node[ground]{}
        (C25_u) -- ++(2,0) to[R, l=$R_{15}$, resistors/scale=0.8] ++(0,-2) node[ground]{};
        %\draw (A |- 52,3)(D3_r) -- (R24_r);
        % to[short] ++(-0.1, 0) -|(R24_r)
    \end{circuitikz}
    \caption{Basic peak detector design.}
    \label{circ:basic_pd}
\end{figure}

Assuming ideal conditions, a diode is enough to retain the highest input voltage reached. Semiconductor diodes however don't behave ideally, they introduce a voltage drop that will keep the voltage stored in $C_{25}$ at a lower potential than that of $V_{in}$. In order to prevent this, an op-amp $(U_6)$\footnote{Currently, the only op-amp that has behaved reasonably well is the LT1807 by Analog Devices Inc. The LMH6658  by Texas Instruments seems to have trouble driving even small capacitors.} is placed before the diode. In the configuration shown in Fig. \ref{circ:basic_pd}, the opamp will try to output the necessary current to equilibrate the inverting input voltage (pin $-$) to what it sees in the non-inverting input (pin $+$), to achieve this $U_6$ has to go one diode drop above $V_{in}$.

\subsection{Preventing negative saturation}\label{sec:pd_V2}

\begin{figure}[H]
    \centering
    \begin{circuitikz}[scale=0.7]
        %\draw [help lines] (-4,0) grid (5,-4);
        \draw (0,0) node[op amp, noinv input up, font=\small] (opamp) {$U_6$}
        (opamp.up) --++(0,0.5) node[vcc]{5\,\textnormal{V}}
        (opamp.down) --++(0,-0.5) node[vee]{-5\,\textnormal{V}}
        (opamp.+) node[left]{$V_{in}$}
        (opamp.out) -- ++(1,0) coordinate(oa_out) to[sD, l_=$D_3$] (6, 0) coordinate (D3_r)
        (opamp.-) -- ++(0, -2.7) coordinate (ver_1) -- ++(1, 0) coordinate(D5_l)
        (ver_1) to[R, l_=$R_{14}$, resistors/scale=0.8] ++(-2,0) node[ground]{} 
        (D5_l) to[sD, l_=$D_5$] (oa_out |- , |- D5_l) coordinate (D5_r)
        (oa_out) -- (D5_r)
        (D5_l) -- ++(0,-1.7) coordinate (R24_l)
        (R24_l) to[R, l^=$R_{24}$, resistors/scale=0.8] (D3_r |- , |- R24_l) coordinate (R24_r)
        (D3_r) -- (R24_r)
        (D3_r) -- ++(1,0) coordinate (C25_u)
        (C25_u) to[C, l=$C_{25}$] ++(0,-2) node[ground]{}
        (C25_u) -- ++(2,0) to[R, l=$R_{15}$, resistors/scale=0.8] ++(0,-2) node[ground]{};
    \end{circuitikz}
    \caption{Basic peak detector design, a second diode is added in order to prevent the op-amp from entering a negative saturation loop.}
    \label{circ:pd_V2}
\end{figure}

In the basic peak detector, once the signal voltage goes below the peak voltage, $D_3$ will be reverse biased and the inverting input of the opamp will see a higher voltage than the non-inverting input, this will force $U_6$ to go into negative saturation by driving the output voltage as low as it can in order to match both inputs. Once the signal gets close to the stored voltage in $C_25$, the op-amp will have to get out of the negative saturation, this will take some time which depends on the slew rate of the opamp and therefore limits the operating frequency range of the circuit.

In order to avoid negative saturation $D_5$ is added, along with an outer feedback loop through $R_{24}$. In this case, once the input signal goes below the stored voltage, $D_5$ will be forward biased, allowing for a new feedback loop that decreases the op-amp's negative saturation time.


\subsection{Basic Peak Detector + Buffer}\label{sec:basic_buffer}

\begin{figure}[H]
    \centering
    \begin{circuitikz}[scale=0.7]
        %\draw [help lines] (-4,0) grid (5,-4);
        \draw (0,0) node[op amp, noinv input up, font=\small] (opamp) {$U_6$}
        (opamp.up) --++(0,0.5) node[vcc]{5\,\textnormal{V}}
        (opamp.down) --++(0,-0.5) node[vee]{-5\,\textnormal{V}}
        (opamp.+) node[left]{$V_{in}$}
        (opamp.out) -- ++(1,0) coordinate(oa_out) to[sD, l_=$D_3$] (6, 0) coordinate (D3_r)
        (opamp.-) -- ++(0, -2.7) coordinate (ver_1) -- ++(1, 0) coordinate(D5_l)
        (ver_1) to[R, l_=$R_{14}$, resistors/scale=0.8] ++(-2,0) node[ground]{} 
        (D5_l) to[sD, l_=$D_5$] (oa_out |- , |- D5_l) coordinate (D5_r)
        (oa_out) -- (D5_r)
        (D5_l) -- ++(0,-1.7) coordinate (R24_l)
        (D3_r) -- ++(1,0) coordinate (C25_u)
        (C25_u) to[C, l=$C_{25}$] ++(0,-2) node[ground]{}
        (C25_u) -- ++(2,0) coordinate (buffer_l)
        (buffer_l) node[op amp, noinv input up, font=\small, anchor=+] (buffer) {$U_4$}
        (buffer.-) coordinate (buffer_-)
        (buffer.out) coordinate (buffer_out)
        (buffer_-) -- (buffer_- |-, |- R24_l) coordinate (R24_r)
        (R24_l) to[R, l^=$R_{24}$, resistors/scale=0.8] (R24_r)
        (R24_r) -- (buffer_out |-, |- R24_r) -- (buffer_out)
        (buffer_out) to[short, -*] ++(1,0);
        %(D3_r) -- (R24_r);
        %(C25_u) -- ++(2,0) to[R, l=$R_{15}$, resistors/scale=0.8] ++(0,-2) node[ground]{};
    \end{circuitikz}
    \caption{Basic peak detector design, Adding a buffer to prevent discharging of the capacitor through the resistor and instead through any load in the circuit, in this case $R_{29}$.}
    \label{circ:pd_buffer}
\end{figure}

This design follows the same principles as the one shown in the previous subsection. However, in this case, a buffer is added to introduce high impedance and prevent the capacitor from discharging through the resistor and instead through any load that may be applied after the circuit.

\subsection{Nuclear Phoenix}\label{sec:nuclear_phoenix}

\begin{figure}[H]
    \centering
    \begin{circuitikz}[scale=0.7]
        %\draw [help lines] (-4,0) grid (5,-4);
        \draw (0,0) node[op amp, noinv input up, font=\small] (opamp) {$U_6$}
        (opamp.up) --++(0,0.5) node[vcc]{5\,\textnormal{V}}
        (opamp.down) --++(0,-0.5) node[vee]{-5\,\textnormal{V}}
        (opamp.+) node[left]{$V_{in}$}
        (opamp.out) coordinate(oa_out) to[sD, l_=$D_5$] ++(2, 0) coordinate (D5_r)
        (D5_r) to[sD, l_=$D_3$] ++(2, 0) coordinate (D3_r)
        (D5_r) -- ++(0,-4) coordinate (R30_l)
        (D3_r) -- ++(3,0) coordinate (C25_u)
        (C25_u) to[C, l=$C_{25}$] ++(0,-2) node[ground]{}
        (C25_u) -- ++(2,0) coordinate (buffer_l)
        (opamp.-) coordinate (oa_-)
        (oa_-) -- ++(-1.5, 0) -- ++(0, 4) coordinate(aux1)
        (aux1) -- (D3_r |-, |- aux1) -- (D3_r)
        (buffer_l) node[op amp, noinv input up, font=\small, anchor=+] (buffer) {$U_4$}
        (buffer.-) coordinate (buffer_-)
        (buffer.out) coordinate (buffer_out)
        (buffer_-) -- (buffer_- |-, |- R30_l) coordinate (R30_r)
        (R30_l) to[R, l_=$R_{30}$, resistors/scale=0.8] (R30_r)
        (R30_r) -- (buffer_out |-, |- R30_r) -- (buffer_out)
        (buffer_out) to[short, -*] ++(1,0);
        %(D3_r) -- (R24_r);
        %(C25_u) -- ++(2,0) to[R, l=$R_{15}$, resistors/scale=0.8] ++(0,-2) node[ground]{};
    \end{circuitikz}
    \caption{Nuclear Phoenix peak-detector design. Taken from \cite{Nucelar_phoenix}.}
    \label{circ:pd_np}
\end{figure}

NuclearPhoenix is a physics student who has developed a gamma detector that also utilizes a Raspberry Pi Pico and a Silicon photomultiplier, his schematics also include a peak detection circuit which is shown in Fig. \ref{circ:pd_np}, his project can be found in \href{https://nuclearphoenix.xyz/hardware/ogd/}{Open Gamma Detector}. This design aims to prevent leakage current across $D_3$, this discharges the capacitor at a faster rate than intended once the peak voltage has been reached. In this case, $R_30$ is feeding back the peak voltage value to $D_3$, therefore creating a 0 V difference across the diode, preventing any leakage current from flowing out of $C_{25}$ and into the output of the op-amp.

\subsection{Nonlinearity}\label{sec:Non-linearity}

The aforementioned nonlinearity between SiPM pulses and the peak detected signal is best showcased in Figure \ref{fig:nonlinearity}. In this case, it is clear that the ADC reading steadily increases up until about 100 \unit{m\V} pulses, after that it starts to plateau, which means that the peak detector is saturating and can no longer differentiate pulse amplitudes. In order to approximate the detector's response, we had to fit an eleven-degree polynomial to the points measured, this means that we would need at least 12 gamma-ray peaks to be able to use this calibration, which in our case has not been possible due to resolution and gamma-sources limitations, for this reason Section \ref{sec:CW_measurements} uses a linear fit to approximate the detector's response. The measured points were taken with the script \texttt{calibration.py}, listed in Appendix \ref{app:calibration}.

\begin{figure}[H]
    \centering
    \includegraphics[width=.98\textwidth]{Electronics/ADC_to_amplitude.pdf}
    \caption{\label{fig:nonlinearity}Calibration of the peak detected signal's ADC reading with respect to SiPM pulse amplitude. The ADC output has been scaled back to twelve bits, as discussed in subsection \ref{sec:ADC_shortcomings}.}
\end{figure}

\section{Trigger}

\begin{figure}[H]
    \centering
    \begin{circuitikz}[scale=0.7]
        %\draw [help lines] (-4,0) grid (5,-4);
        \draw (0,0) node[op amp, noinv input up, font=\small] (opamp) {$U_4$}
        (opamp.up) --++(0,0.5) node[vcc]{5\,\textnormal{V}}
        (opamp.down) --++(0,-0.5) node[vee]{-5\,\textnormal{V}}
        (opamp.out) --++(1,0) node[above]{$V_{out}$}
        (opamp.+) node[left]{$V_{in}$}
        (opamp.-) --++(-2.5,0)  coordinate (oa_-)
        (oa_-) to[R, l^=$R_{16}$, resistors/scale=0.8] ++(0,2) to[R, l^=$R_{22}$, resistors/scale=0.8] ++(0,2) node[vcc]{5\,\textnormal{V}}
        (oa_-) to[R, l_=$R_{18}$, resistors/scale=0.8] ++(0,-2) node[ground]{};

        %Nodes
        \node[shift={(-0.3,-0.3)}] at (opamp.-) {$V_{REF}$};
    \end{circuitikz}
    \caption{Trigger circuit.}
    \label{circ:trigger}
\end{figure}

In this case, a voltage divider is used to force a positive saturation in the op-amp once the amplified signal reaches a threshold voltage, generating a "digital 1"\: that is the used to trigger an ADC reading of the peak detected signal. The threshold, or $V_{REF}$ as noted in Fig. \ref{circ:trigger}, is given by equation \eqref{eq:v_ref}.

\begin{equation}
    V_{REF}=\frac{R_{18}}{R_{22}+R_{16}+R_{18}} V_{cc} \label{eq:v_ref}
\end{equation}

%add code as appendix
\section{Microcontroller}

The core of this project lies on the RaspberryPi Pico microcontroller, its two cores allow us to acquire SiPM pulse data while simultaneously saving it to a file and display it on the OLED screen along with time, count rate, temperature and pressure information. In addition to this, the Pico can be controlled with MicroPython, a smaller version of the Python3 programming language, which can make the code easier to understand and modify by new CosmicWatch users, specially students who might be new to electronics.

\begin{figure}
    \centering
    \includegraphics[width=0.9\textwidth]{Electronics/RP_Schematic.pdf}
    \caption{RaspberryPi Pico connections in the detector.}
    \label{fig:RP_schematic}
\end{figure}

Figure \ref{fig:RP_schematic} shows the Pico connections in the setup used for this work, we will do a small introduction to the general functionalities of these connections.

\begin{itemize}
    \item \textbf{2-7}: Control the MicroSD card socket which uses the Serial Peripheral Interface (SPI) communication protocol.
    \item \textbf{16-17,36}: Pins 16 and 17 are used to display information on the OLED screen in addition to measure pressure and temperature data from the BMP280 sensor using the I2C communication protocol, each accessory has a different address that enables the use of multiple devices through the same connection. Power for these accessories is provided by pin 36
    \item \textbf{19-20}: The LEDs that signal single and coincident events are connected to these pins, resistors $R_{37}$ and $R_{17}$ can be changed to reduce power consumption.
    \item \textbf{21 and 27}: Pin 21 is connected to the output of the trigger circuit, in the code this activates the \texttt{read\_ADC} interrupt-routine which makes an ADC reading of the peak detected signal, with this we aim at reducing the error in the voltage measured due to the discharge of capacitor $C_{25}$. \\Pin 27 is controlled by the code and is in charge of resetting the trigger, this is achieved by fully discharging $C_{25}$ through the mosfet labeled $Q1$ in the schematic, thus preventing the accumulation of charge from various pulses that may arrive in a time window smaller than the natural discharging of the capacitor.
    \item \textbf{25-26}: The coincident pins are used by the \texttt{CoincidentMode} function in \texttt{run.py} to determine the role of the detector when it runs in coincidence mode.
    \item \textbf{30}: This pin is connected to the reset button, once it is pressed the detector stops any tasks it might be running and will automatically run the \texttt{main.py} script.
    \item \textbf{31}: All ADC readings of the peak detected signal are done through this pin.
    \item \textbf{39-40}: The circuit connected to these pins is designed to produce a low noise power supply for the various integrated circuits used in this project, $D_1$ in particular is placed in a way that prevents current from flowing into the Pico.
\end{itemize}

\section{DC to DC booster} \label{sec:DC_DC}

\begin{figure}[H]
    \centering
    \begin{circuitikz}
        % U1 NE555
        \draw [thick] (0,0) coordinate (u1) rectangle ++(2,3); % shape
        \draw [pin] (u1) ++(1, 3.8)
            node[]{$U_1$};
        \draw [pin] (u1) ++(1, 3.3)
            node[]{MAX5026};
        %-----------Left side-----------%
        \draw [pin] (u1) ++ (0,2.5) coordinate (u1 pgnd)
            node[right, font=\scriptsize]{PGND}
            node[above left]{1}; % CON
        \draw [pin] (u1) ++ (0,1.5) coordinate (u1 gnd)
            node[right, font=\scriptsize]{GND}
            node[above left]{2}; % TRI
        \draw [pin] (u1) ++ (0,0.5) coordinate (u1 fb)
            node[right, font=\scriptsize]{FB}
            node[above left]{3}; % THR
        
        %-----------Right side-----------%
        \draw [pin] (u1) ++ (2,2.5) coordinate (u1 lx)
            node[left, font=\scriptsize]{LX}
            node[above right]{6}; % OUT
        \draw [pin] (u1) ++ (2,1.5) coordinate (u1 vcc)
            node[left, font=\scriptsize]{VCC}
            node[above right]{5}; % OUT
        \draw [pin] (u1) ++ (2,0.5) coordinate (u1 shdn)
            node[left, font=\scriptsize]{SHDN}
            node[above right]{4}; % OUT
        %\draw (u1) ++ (2,0)
        %    node[right]{\ctikzlabel{$U_1$}{NE555}}; % NE555P
        
        % U1 NE555 Pins
        \draw (u1 pgnd) -- ++ (-1,0) node[ground]{}; % PGND

        \draw (u1 gnd) -- ++ (-1,0) node[ground]{}; % GND

        \draw (u1 fb) -- ++ (-2,0) coordinate(r10 u) to[R, l_=$R_{10}$, resistors/scale=0.8] ++(0,-2) node[ground]{}
        (r10 u) to[R, l^=$R_{2}$, resistors/scale=0.8] ++(0,2) -- ++(0,2) -- ++(7,0) coordinate(d4 r)
        (d4 r) --++(2,0) coordinate(c3 u) to[R, l^=$R_{25}$, resistors/scale=0.8] ++(2,0) coordinate (c4 u)
        (c3 u) to[C, l_=$C_{3}$, resistors/scale=0.8] ++(0,-2)  node[ground]{}
        (c4 u) to[C, l_=$C_{4}$, resistors/scale=0.8] ++(0,-2)  node[ground]{}
        (c4 u) --++(1,0) node[vcc]{\textnormal{HV}}; % FB

        \draw (u1 lx) -- ++ (1,0) coordinate(d4 l)
        (d4 l) -- ++(0,1) to[sD, l^=$D_3$] ($(d4 r) + (0,-1)$) -- (d4 r)
        (d4 l) to[american inductor, l_=$L_1$] ($(d4 r) + (0,-2)$) coordinate(l1 r); % LX

        \draw (u1 vcc) -- ++ (1,0) to[short, -*] ++(0,-1); % VCC

        \draw (u1 shdn) --++(2,0) coordinate(c7 u) --++(1,0) coordinate (c2 u) --(l1 r)
        (c7 u) to[C, l_=$C_{7}$, resistors/scale=0.8] ++(0,-2) node[ground]{}
        (c2 u) to[C, l=$C_{2}$, resistors/scale=0.8] ++(0,-2) node[ground]{}
        (c2 u) --++(1,0) node[vcc]{5\,\textnormal{V}}; % SHDN
    \end{circuitikz}
    \caption{DC to DC booster circuit. Careful considerations have to be made when placing the MAX5026 IC in order to reduce noise on the power line. It is advised to have a look at the datasheet in \cite{AnalogDevices_DC_DC}, section "Applications Information".}
    \label{circ:booster}
\end{figure}

According to Onsemi \cite{Onsemi_SiPM}, the operating voltage of a MicroFJ-300XX-TSV SiPM lies between 25.2 and 30.7 V. It is then necessary to boost $V_{cc}$ to the operating range of the SiPM, noted as $HV$ in Fig. \ref{circ:booster}. In order to achieve this, a MAX5026 PWM Step-Up DC-DC Converter is used \cite{AnalogDevices_DC_DC}, which has a user-adjustable output voltage of up to 36 V using external feedback resistors.

In the circuit shown in Fig. \ref{circ:booster}, $R_{10}$ and $R_2$ determine the output voltage $HV$. According to Ref. \cite{AnalogDevices_DC_DC}, equation \eqref{eq:booster_fb_resistance} allows to calculate the Value of $R_{10}$ given a desired output voltage $HV$ and a value of $R_2$ between 5 and 50 \unit{\kilo\ohm}.

\begin{equation}
    R_{10} = R_2\left(\frac{HV}{V_{REF}}-1\right) \label{eq:booster_fb_resistance}
\end{equation}