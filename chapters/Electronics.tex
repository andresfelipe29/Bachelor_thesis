\chapter{Electronics}

CosmicWatches have to be mainly low-cost and easy to build, in order to achieve this the components selected for the construction have been carefully curated to make sure these restrictions were met. This however might be greatly responsible for some of the odd features found while testing the detector, like the lack of linearity. The full KiCad project can be found in the GitHub repository: \href{https://github.com/anvargasl/CosmicWatch-gamma-spectroscopy-PCB}{CosmicWatch-gamma-spectroscopy-PCB}. The component numbers shown in this section are the ones that would have to be placed on the PCB in order to recreate the example schematics.

\section{Amplifier}

An op-amp on its own amplifies the voltage difference between the non-inverting (pin $+$) and inverting (pin $-$) inputs by its internal gain $A_{int}$, having then $V_{out}=A_{int}(V_+ - V_-)$. In this case, however, we are interested in controlling the gain of the circuit and therefore the amplification. In order to achieve this we introduce a feedback loop in the op-amp through $R4$ and $R6$, which controls how much of the output voltage is fed back into the op-amp. The theoretical amplification is therefore given by $V_{out}=(1+R6/R4)V_{in}$. A simple schematic showcasing the component arrangement is shown in Fig. \ref{circ:amplifier}.
\begin{figure}[H]
    \centering
    \begin{circuitikz}[scale=0.8]
        \draw
        (0,0) node[op amp, noinv input up, font=\small] (opamp) {$U6$}
        (opamp.up) --++(0,0.5) node[vcc]{-5\,\textnormal{V}}
        (opamp.down) --++(0,-0.5) node[vee]{5\,\textnormal{V}}
        (opamp.+) node[left]{$V_{in}$}
        (opamp.-) node[left]{} to[short] ++(0, -3) coordinate(R4_R)
        to[R, l_=$R_4$] ++(-3, 0) node[ground]{}
        (R4_R) to[R, l=$R_6$, *-] ++(2.7, 0)
        %to[short] ++(0.1, 0) -|(opamp.out) to[short] ++(1, 0) node[ocirc, label={[yshift=0.3]$V_{out}$}]{}
        to[short] ++(0.1, 0) -|(opamp.out) to[short] ++(1, 0) node[ocirc, label=above:$V_{out}$]{};

        %Nodes
        %\node[shift={(0,+1.5)}] at (opamp) {U6 HLM6658};
    \end{circuitikz}
    \caption{Amplifier circuit schematic. An LT1807 op-amp is used for this and the peak detection stage.}
    \label{circ:amplifier}
\end{figure}

\section{Peak Detector}

The idea behind the peak detector is to store charge in a capacitor ($C_{25}$) through a diode ($D_3$), retaining the highest voltage the input signal reaches. A diode is placed before the capacitor so that once the signal's voltage goes below the peak voltage, the diode will be reverse biased, therefore preventing current from flowing while maintaining the voltage on the capacitor.

In order to measure the voltage in the capacitor, a discharging resistor has to be added ($R_{15}/R_{19}$). The time it takes the capacitor to discharge is given by $t=RC$. Although for example in the case of CosmicWatch-V2's peak detector, the values of $R_{14}$ and $R_{24}$ also play a role in the discharging time, which has proved not to be as trivial as calculating the equivalent resistance $R_t$ of all three and simply take $t=R_tC$.

The schematic and PCB shown in the repository \href{https://github.com/anvargasl/CosmicWatch-gamma-spectroscopy-PCB}{CosmicWatch-gamma-spectroscopy-PCB}, include the connections and footprints necessary to place the components that make the designs illustrated in Subsections \labelcref{sec:basic,sec:pd_V2,sec:basic_buffer,sec:nuclear_phoenix}. Different results were found while testing these peak detector setups.

\subsection{Basic Peak Detector}\label{sec:basic}

\begin{figure}[H]
    \centering
    \begin{circuitikz}[scale=0.8]
        %\draw [help lines] (-4,0) grid (5,-4);
        \draw (0,0) node[op amp, noinv input up, font=\small] (opamp) {$U6$}
        (opamp.up) --++(0,0.5) node[vcc]{-5\,\textnormal{V}}
        (opamp.down) --++(0,-0.5) node[vee]{5\,\textnormal{V}}
        (opamp.+) node[left]{$V_{in}$}
        (opamp.out) node[]{} to[sD, l_=$D_3$] (3.5, 0) coordinate (D3_r)
        (opamp.-) node[left]{} to[short] ++(0, -3) coordinate (R24_l)
        (R24_l) to[R, l^=$R_{24}$, resistors/scale=0.8] (D3_r |- , |- R24_l) coordinate (R24_r)
        (D3_r) -- (R24_r)
        (D3_r) -- ++(1,0) coordinate (C25_u)
        (C25_u) to[C, l=$C_{25}$] ++(0,-2) node[ground]{}
        (C25_u) -- ++(2,0) to[R, l=$R_{15}$, resistors/scale=0.8] ++(0,-2) node[ground]{};
        %\draw (A |- 52,3)(D3_r) -- (R24_r);
        % to[short] ++(-0.1, 0) -|(R24_r)
    \end{circuitikz}
    \caption{Basic peak detector design.}
    \label{circ:basic_pd}
\end{figure}

Assuming ideal conditions, a diode is enough to retain the highest input voltage reached. Semiconductor diodes however don't behave ideally, they introduce a voltage drop that will keep the voltage stored in $C_{25}$ at a lower potential than that of $V_{in}$. In order to prevent this, an op-amp $(U_6)$\footnote{Currently the only op-amp that has behaved reasonably well is the LT1807 by Analog Devices Inc. The LMH6658  by Texas Instruments seems to have trouble driving even small capacitors.} is placed before the diode. In the configuration shown in Fig. \ref{circ:basic_pd}, the opamp will try to output the necessary current to equilibrate the inverting input voltage (pin $-$) to what it sees in the non-inverting input (pin $+$), to achieve this $U_6$ has to go one diode drop above $V_{in}$.

\subsection{Preventing negative saturation}\label{sec:pd_V2}

\begin{figure}[H]
    \centering
    \begin{circuitikz}[scale=0.8]
        %\draw [help lines] (-4,0) grid (5,-4);
        \draw (0,0) node[op amp, noinv input up, font=\small] (opamp) {$U6$}
        (opamp.up) --++(0,0.5) node[vcc]{-5\,\textnormal{V}}
        (opamp.down) --++(0,-0.5) node[vee]{5\,\textnormal{V}}
        (opamp.+) node[left]{$V_{in}$}
        (opamp.out) coordinate(oa_out) to[sD, l_=$D_3$] (3.5, 0) coordinate (D3_r)
        (opamp.-) -- ++(0, -3) coordinate (ver_1) -- ++(1, 0) coordinate(D5_l)
        (ver_1) to[R, l_=$R_{14}$, resistors/scale=0.8] ++(0,-2) node[ground]{} 
        (D5_l) to[sD, l_=$D_5$] (oa_out |- , |- D5_l) coordinate (D5_r)
        (oa_out) -- (D5_r)
        (D5_l) -- ++(0,-2) coordinate (R24_l)
        (R24_l) to[R, l_=$R_{24}$, resistors/scale=0.8] (D3_r |- , |- R24_l) coordinate (R24_r)
        (D3_r) -- (R24_r)
        (D3_r) -- ++(1,0) coordinate (C25_u)
        (C25_u) to[C, l=$C_{25}$] ++(0,-2) node[ground]{}
        (C25_u) -- ++(2,0) to[R, l=$R_{15}$, resistors/scale=0.8] ++(0,-2) node[ground]{};
    \end{circuitikz}
    \caption{Basic peak detector design, a second diode is added in order to prevent the op-amp from entering a negative saturation loop.}
    \label{circ:pd_V2}
\end{figure}

\subsection{Basic Peak Detector + Buffer}\label{sec:basic_buffer}

\begin{figure}[H]
    \centering
    \begin{circuitikz}[scale=0.8]
        %\draw [help lines] (-4,0) grid (5,-4);
        \draw (0,0) node[op amp, noinv input up, font=\small] (opamp) {$U6$}
        (opamp.up) --++(0,0.5) node[vcc]{5\,\textnormal{V}}
        (opamp.down) --++(0,-0.5) node[vee]{-5\,\textnormal{V}}
        (opamp.+) node[left]{$V_{in}$}
        (opamp.out) coordinate(oa_out) to[sD, l_=$D_3$] (3.5, 0) coordinate (D3_r)
        (opamp.-) -- ++(0, -3) coordinate (ver_1) -- ++(1, 0) coordinate(D5_l)
        (ver_1) to[R, l_=$R_{14}$, resistors/scale=0.8] ++(0,-2) node[ground]{} 
        (D5_l) to[sD, l_=$D_5$] (oa_out |- , |- D5_l) coordinate (D5_r)
        (oa_out) -- (D5_r)
        (D5_l) -- ++(0,-2) coordinate (R24_l)
        (D3_r) -- ++(1,0) coordinate (C25_u)
        (C25_u) to[C, l=$C_{25}$] ++(0,-2) node[ground]{}
        (C25_u) -- ++(2,0) coordinate (buffer_l)
        (buffer_l) node[op amp, noinv input up, font=\small, anchor=+] (buffer) {$U7$}
        (buffer.-) coordinate (buffer_-)
        (buffer.out) coordinate (buffer_out)
        (buffer_-) -- (buffer_- |-, |- R24_l) coordinate (R24_r)
        (R24_l) to[R, l_=$R_{24}$, resistors/scale=0.8] (R24_r)
        (R24_r) -- (buffer_out |-, |- R24_r) -- (buffer_out)
        (buffer_out) -- ++(2,0) to[R, l=$R_{29}$, resistors/scale=0.8] ++(0,-2) node[ground]{};
        %(D3_r) -- (R24_r);
        %(C25_u) -- ++(2,0) to[R, l=$R_{15}$, resistors/scale=0.8] ++(0,-2) node[ground]{};
    \end{circuitikz}
    \caption{Basic peak detector design, Adding a buffer to prevent discharging of the capacitor through the resistor and instead through any load in the circuit, in this case $R_29$.}
    \label{circ:pd_buffer}
\end{figure}

\subsection{Nuclear Phoenix}\label{sec:nuclear_phoenix}

\begin{figure}[H]
    \centering
    \begin{circuitikz}[scale=0.8]
        %\draw [help lines] (-4,0) grid (5,-4);
        \draw (0,0) node[op amp, noinv input up, font=\small] (opamp) {$U6$}
        (opamp.up) --++(0,0.5) node[vcc]{5\,\textnormal{V}}
        (opamp.down) --++(0,-0.5) node[vee]{-5\,\textnormal{V}}
        (opamp.+) node[left]{$V_{in}$}
        (opamp.out) coordinate(oa_out) to[sD, l_=$D_5$] ++(2, 0) coordinate (D5_r)
        (D5_r) to[sD, l_=$D_3$] ++(2, 0) coordinate (D3_r)
        (D5_r) -- ++(0,-4) coordinate (R24_l)
        (D3_r) -- ++(3,0) coordinate (C25_u)
        (C25_u) to[C, l=$C_{25}$] ++(0,-2) node[ground]{}
        (C25_u) -- ++(2,0) coordinate (buffer_l)
        (opamp.-) coordinate (oa_-)
        (oa_-) -- ++(-1.5, 0) -- ++(0, 4) coordinate(aux1)
        (aux1) -- (D3_r |-, |- aux1) -- (D3_r)
        (buffer_l) node[op amp, noinv input up, font=\small, anchor=+] (buffer) {$U7$}
        (buffer.-) coordinate (buffer_-)
        (buffer.out) coordinate (buffer_out)
        (buffer_-) -- (buffer_- |-, |- R24_l) coordinate (R24_r)
        (R24_l) to[R, l_=$R_{24}$, resistors/scale=0.8] (R24_r)
        (R24_r) -- (buffer_out |-, |- R24_r) -- (buffer_out)
        (buffer_out) -- ++(2,0) to[R, l=$R_{29}$, resistors/scale=0.8] ++(0,-2) node[ground]{};
        %(D3_r) -- (R24_r);
        %(C25_u) -- ++(2,0) to[R, l=$R_{15}$, resistors/scale=0.8] ++(0,-2) node[ground]{};
    \end{circuitikz}
    \caption{Nuclear Phoenix peak-detector desing. Taken from \cite{Nucelar_phoenix}.}
    \label{circ:pd_np}
\end{figure}

\section{Trigger}

%add code as appendix
\section{Microcontroller}

\section{DC to DC booster}

\section{Single photons}