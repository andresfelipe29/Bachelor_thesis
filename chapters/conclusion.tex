\chapter{Conclusions}

\begin{itemize}
    \item CosmicWatch: The Desktop Muon Detectors show promising potential in gamma spectroscopy. The integration of Silicon photomultipliers with LYSO crystals represents a significant improvement from previous versions that used plastic scintillators.\item Despite technical challenges, it was proven that the detector is capable of resolving peaks, this progress was achieved through meticulous optimizations in hardware and software, marking a substantial advancement towards a fully functional gamma spectrometer, achieving an energy resolution of 36.5\% at 511 keV.
    \item The multithreading approach, made possible with the use of a Raspberry Pi Pico, enabled the detector to keep up with the high activities of our radioactive sources, significantly decreasing dead time and therefore boosting its sample rate.
    \item It was shown that our electronics can be coupled with other acquisition systems, such as the Rohde\&Schwarz RTO6 oscilloscope and NIM modules, providing flexibility and opening the door for accessible and user-friendly detectors.
    \item Further research needs to be done in order to take full advantage of the ADC and improve linearity in the electronics. The implementation of the \texttt{C/C++} SDK is a promising first approach.
\end{itemize}

The enhancements to CosmicWatch were obtained while remaining affordable and user-friendly, aligning with our primary goal of keeping science available to students, educators, researchers, and enthusiasts alike, inviting them to delve into the world of particle and nuclear physics, electronics, programming, and data analysis. Furthermore, we outline plausible paths to build upon our work and encourage CosmicWatch users to use these materials as a stepping stone to further contribute to the project.