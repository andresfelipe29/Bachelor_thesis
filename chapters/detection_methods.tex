\chapter{Detection methods} \label{chap:detection_methods}

%versions of the detector
\section{Scintillation}

\section{Single photon detectors}

\section{PMTs}

Photomultipliers are very useful tools in particle detection. Since some detection methods rely on the measurement of photons it is very important to make use of every single one. Some PMs are capable of creating measurable electrical signals from single photon interactions, provoking an electron avalanche that can be easily read with various devices. Photomultiplier Tubes are the most common light amplifiers paired with scintillators, they are however bulky and require very high voltages to operate. An in-depth description of this and other types of Photomultipliers can be found in Ref. \cite{knoll2010radiation} Chap. 9.

\section{SiPM advantages}

CosmicWatch is meant to be a portable, easy-to-build, and economical device, making PMTs not suitable for the overall intention of the project. This, however, is fully solved by Silicon Photomultipliers, their small form factor, relatively low operating voltage, and good efficiency at wavelengths near the emission maxima of common scintillating materials, turn them into an ideal candidate for use in this project. For further details on SiPMs also read Ref. \cite{knoll2010radiation} Chap. 9 and Ref. \cite{Onsemi_SiPM_intro}.

A Silicon Photomultiplier is an array of Geiger-mode avalanche-photodiodes. To make this more digestible we can introduce these concepts one by one. A photodiode is essentially made of a P-N semiconductor junction, where scintillation photons can create new electron-hole pairs that will be accelerated by the bias voltage applied to the diode. However, this process alone produces small signals, making a preamplification stage necessary. Avalanche photodiodes alleviate this problem by increasing the bias voltage, therefore giving the photoelectrons higher energies, allowing them to collide and produce new electron-hole pairs that will also be detected. When the bias voltage rises sufficiently high, the regions where photoelectrons multiply merge together, making one big avalanche, this is the so-called Geiger mode of APDs, and this high voltage receives the name of Breakdown Voltage. Geiger mode APDs can produce a large output from a single scintillation photon. The difference Between Bias and Breakdown Voltage is known as Over-voltage.

SiPMs however, also have some disadvantages, since they are based on the transport of photoelectrons, the output can vary with temperature, as high temperatures can create electron-hole pairs that will be read as a photon interaction. These events are called dark pulses and represent a random noise in the signal.

The SiPM used and tested throughout this work is a MicroFJ-300XX-TSV by Onsemi \cite{Onsemi_SiPM} (Fig. \ref{fig:Onsemi_SiPM}), which has an active area of $3.07 \times 3.07$ \unit{\mm\squared} and a microcell size of 35 \unit{\micro\m}. In this case, the operating voltage ranges between 25.2 and 30.7 \unit{\V}, which can be obtained using a DC to DC booster (see Section \ref{sec:DC_DC}), no longer requiring the high voltages of PMTs. We are currently operating at an Over-voltage of 4 V, which according to Ref. \cite{Onsemi_SiPM} produces a relatively high dark count rate (50-150 \unit{\kilo\Hz\per\mm\squared} at 21 $^\circ$C), potentially affecting energy resolution.

\begin{figure}[H]
    \centering
    \includegraphics[width=0.8\textwidth]{detection_methods/MicroFJ−300XX−TSV.jpeg}
    \caption{Onsemi MicroFJ-300XX-TSV. Taken from \cite{Onsemi_SiPM}.}
    \label{fig:Onsemi_SiPM}
\end{figure}