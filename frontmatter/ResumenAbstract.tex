\newpage
\chapter*{Resumen}\chaptermark{Resumen}
\addcontentsline{toc}{chapter}{Resumen}%
\textbf{\Huge \ \strut CosmicWatch: Los Detectores de Muones de Escritorio, explorando la espectroscopia gamma \strut} %strut keeps interline spacing consistent
\vspace{.5cm}
\par El presente trabajo se concentra en el mejoramiento de CosmicWatch: Los Detectores de Escritorio de Muones, con el objetivo de obtener una resolución en energías suficiente para realizar espectroscopia gamma. Dadas ciertas limitaciones actuales en la electrónica, como tiempo muerto y baja rata de conteo, junto con la baja resolución de los centelladores plásticos usados hasta ahora, este trabajo explora entonces el uso de una RaspberryPi Pico y un cristal centellador basado en Lutecio y dopado con Cerio (LYSO:Ce), logrando resolución en energías de $13.5\%$ para 511 \unit{\kilo\eV} al medir con un osciloscopio Rohde\&Schwarz RTO6, $28.3\%$ con módulos NIM, y 36.5\% al procesar la señal con nuestros diseños de amplificación y detección de picos.
\\[2cm]
\textbf{Palabras clave:} \palabrasclave

\newpage 
\chapter*{Abstract}\chaptermark{Abstract}
\addcontentsline{toc}{chapter}{Abstract}%
\textbf{\Huge \ \strut CosmicWatch: The Desktop Muon Detectors, exploring gamma-ray spectroscopy \strut} %strut keeps interline spacing consistent
\vspace{.5cm}
\par The present work focuses on the improvement of CosmicWatch: The Desktop Muon Detectors, with the goal of achieving sufficient energy resolution to allow for gamma spectroscopy. Due to current limitations in the electronics, such as dead time and small sample rate, in addition to the low energy resolution of previously used plastic scintillators, this work explores the use of a RaspberryPi Pico together with a Cerium-doped Lutetium-based scintillation crystal (LYSO:Ce), achieving energy resolutions of $13.5\%$ at 511 \unit{\kilo\eV} while sampling data with a Rohde\&Schwarz RTO6 oscilloscope, $28.3\%$ with NIM modules, and 36.5\% while using our amplifier and peak-detector designs.
\\[2cm]
\textbf{Keywords:} \keywords

%\newpage 
%\chapter*{\sffamily Zusammenfassung}
%\addcontentsline{toc}{chapter}{Zusammenfassung}%
%\par Zusammenfassung texte.
%\par 
%\\[2cm]
%\textbf{Schlüsselwörter:} \schlusselworter