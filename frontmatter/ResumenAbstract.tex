\newpage
\chapter*{Resumen}\chaptermark{Resumen}
\addcontentsline{toc}{chapter}{Resumen}%
\textbf{\Huge \ \strut CosmicWatch: Los Detectores de Muones de Escritorio, explorando la espectroscopia gamma \strut} %strut keeps interline spacing consistent
\vspace{.5cm}
\par El presente trabajo se concentra en el mejoramiento de CosmicWatch: Los Detectores de Muones de Escritorio, con el objetivo de evolucionar de un detector tipo contador Geiger a un espectrómetro de gammas completamente funcional. Requiriendo por lo tanto buena resolución de energías y altas velocidades de sampleo, siendo estas las principales limitaciones de versiones anteriores de CosmicWatch. Usando un cristal centellador basado en Lutecio y dopado con Cerio (LYSO), se logró una resolución de $4.86 \sqrt{E~[\unit{\MeV}]}$ al samplear datos con un osciloscopio Rohde\&Schwarz RTO6.
\\[2cm]
\textbf{Palabras clave:} \palabrasclave

\newpage 
\chapter*{Abstract}\chaptermark{Abstract}
\addcontentsline{toc}{chapter}{Abstract}%
\textbf{\Huge \ \strut CosmicWatch: The Desktop Muon Detectors, exploring gamma-ray spectroscopy \strut} %strut keeps interline spacing consistent
\vspace{.5cm}
\par The present work focuses on the improvement of CosmicWatch: The Desktop Muon Detectors, with the goal to evolve from a Geiger-counter type of detector to a fully functional gamma-spectrometer. Therefore requiring good energy resolution and fast sample rates, which were the main limitations of previous CosmicWatch versions. Using a Cerium doped Lutetium-based scintillation crystal (LYSO), we have achieved an energy resolution of $4.86 \sqrt{E~[\unit{\MeV}]}$ while sampling data with a Rohde\&Schwarz RTO6 oscilloscope.
\\[2cm]
\textbf{Keywords:} \keywords

%\newpage 
%\chapter*{\sffamily Zusammenfassung}
%\addcontentsline{toc}{chapter}{Zusammenfassung}%
%\par Zusammenfassung texte.
%\par 
%\\[2cm]
%\textbf{Schlüsselwörter:} \schlusselworter