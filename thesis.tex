% Esta es la Plantilla UNAL en LaTeX
\documentclass[12pt,spanish,fleqn,openany,twoside,letterpaper]{book}

%Muestra los márgenes del documento para evitar Warnings
%Para activar la siguiente línea quite el símbolo % 
%\usepackage[showframe]{geometry}

%Formato de fuentes bibliográficas
%Use el estilo bibliográfico que sea pertinente según el área de estudio APA, IEEE, etc

%Usando el paquete BibLaTeX
%Cita normal con \cite[página]{} y cita con paréntesis \parencite[página]{}

% Configuración de BibLaTeX
%\usepackage[backend=biber,style=authoryear,maxcitenames=2,maxbibnames=99,giveninits=true,uniquename=false]{biblatex}
%\addbibresource{Referencias.bib}

% Cambiar el idioma de las referencias bibliográficas a español
%\DefineBibliographyStrings{spanish}{%ext install latex-workshop
%  andothers = {et\addabbrvspace al\adddot},
%  andmore = {et\addabbrvspace al\adddot},
%}

% Personalizar el formato de las citas y la bibliografía
%\DeclareNameAlias{sortname}{family-given}
%\DeclareDelimFormat{multinamedelim}{\addcomma\space}
%\DeclareDelimFormat{finalnamedelim}{\addcomma\space\&\space}
%\DeclareFieldFormat{titlecase}{\MakeSentenceCase*{#1}}
%\DeclareFieldFormat[article,inbook,incollection,inproceedings,patent,thesis,unpublished]{title}{\titlecase{#1}}
%\DeclareFieldFormat{journaltitlecase}{\titlecase{#1}}
%\DeclareFieldFormat{pages}{#1}
%\DeclareFieldFormat{volume}{\mkbibbold{#1}}
%\renewbibmacro{in:}{}
%\AtEveryBibitem{\clearfield{month}}

%Usando el paquete Natbib
%Cita normal \cite[página]{} y cita con paréntesis \citep[página]{}
%\usepackage{natbib}
%\bibpunct{[}{]}{;}{\&}{.}{}
%\bibliographystyle{dtvstyle}

\usepackage[nottoc]{tocbibind} %------bibliografía------%

%Idioma del documento
%Use main para el idioma principal del documento
\usepackage[main=spanish,english,german,french,portuguese]{babel}

% Caracteres especiales
\usepackage{fontenc}

% Evita ligadura li & fl
\usepackage{microtype}
\DisableLigatures{encoding = *, family = *}

% Otros paquetes de tablas y colores avanzados
\usepackage{amsmath,graphicx,rotating,float,multirow}
\usepackage{caption}
\usepackage{longtable}
\setlength{\LTcapwidth}{6in}
\usepackage[utf8]{inputenc}
\usepackage{epsfig,epic,eepic,threeparttable,amscd,here,lscape,tabularx,subfigure}
\usepackage{tabu,array}
\usepackage[rgb]{xcolor}

\usepackage{siunitx} %------unidades------%

% Permite ver y configurar los parámetros de la página
\usepackage{layout}
%Hyperref permite ver las secciones del texto
\usepackage[hidelinks]{hyperref}
\hypersetup{
    colorlinks=true,
    linkcolor=blue,
    filecolor=magenta,      
    urlcolor=blue,
    citecolor=blue,
}
\usepackage{cleveref} %cleveref has to be added after hyperref

\crefrangelabelformat{subsection}{#3#1#4-#5\crefstripprefix{#1}{#2}#6}

%Genera los comandos de la página de autoría
\newcommand{\studentname}{}
\newcommand{\submissiondate}{}
\newcommand{\academictitle}{}
\newcommand{\resgroupone}{}
\newcommand{\resgrouptwo}{}
\newcommand{\researchtopic}{}
\newcommand{\thesisname}{}
\newcommand{\director}{}
\newcommand{\codirector}{}
\newcommand{\issuedate}{}
\newcommand{\palabrasclave}{}
\newcommand{\keywords}{}
\newcommand{\schlusselworter}{}
\newcommand{\palavraschave}{}
\newcommand{\sede}{}
\newcommand{\department}{}
\newcommand{\faculty}{}

%Información de la tesis
%Diligenciar aquí los datos para su carga automática donde se requiera en el documento
\renewcommand{\studentname}{Andrés Felipe Vargas-Londoño}
\renewcommand{\thesisname}{CosmicWatch: The Desktop Muon Detectors, exploring gamma-ray spectroscopy}
\renewcommand{\issuedate}{2024}
\renewcommand{\submissiondate}{Fecha entrega}
\renewcommand{\director}{Prof. Luis Fernando Cristancho Mejia}
\renewcommand{\codirector}{Prof. Spencer Axani}
\renewcommand{\academictitle}{Físico}
\renewcommand{\resgroupone}{Grupo de Física Nuclear Universidad Nacional (GFNUN)}
\renewcommand{\resgrouptwo}{Axani Group (AxLab)}
\renewcommand{\researchtopic}{Espectroscopía gamma}
\renewcommand{\sede}{Sede Bogotá} 
\renewcommand{\department}{Departamento de Física}
\renewcommand{\faculty}{Facultad de Ciencias}

%Palabras clave del documento - Tener presente los Theasurus https://www.thesaurus.com/
%Disponible en 3 idiomas aunque se puede extender a francés o otro idioma
\renewcommand{\palabrasclave}{Use palabras clave que estén en Theasaurus} 
\renewcommand{\keywords}{Use keywords available in Theasaurus}
%\renewcommand{\schlusselworter}{}
%\renewcommand{\palavraschave}{}

%Formatting for headers and footers
\usepackage{fancyhdr}
\pagestyle{fancy}

\fancypagestyle{plain}

\newcommand{\RomanNumeralCaps}[1]
    {\MakeUppercase{\romannumeral #1}}

%\fancyhead[RO,LE]{Monolithic Nanocomposite Detector for LaBrAT-PET}

% Redefinir \sectionmark
\renewcommand{\sectionmark}[1]{\markright{#1}}
% Redefinir \leftmark para mostrar solo el nombre de la sección
\renewcommand{\chaptermark}[1]{\markboth{#1}{}}

% Configurar encabezado y pie de página
\fancyhead[L]{\fontsize{10}{12} \selectfont Personal Statement}
\fancyhead[RO]{\fontsize{10}{12} \selectfont \thepage} % Encabezado derecho en páginas impares
\fancyhead[LO]{\fontsize{10}{12} \selectfont \textbf{\leftmark}}   % Encabezado izquierdo en páginas impares
\fancyhead[RE]{\fontsize{10}{12} \selectfont \textbf{\thesisname}}   % Encabezado izquierdo en páginas pares
\fancyhead[LE]{\fontsize{10}{12} \selectfont \thepage}  % Encabezado derecho en páginas pares
\fancyfoot[C]{}            % Pie de página central vacío

\usepackage{titlesec}
% Permite personalizar los títulos de sección y de capítulos
% hang lo deja en el mismo renglón, display lo despliega
% Elimina el "Capitulo" y deja solo el número
%\titleformat{\chapter}[hang]
%  {\sffamily\Huge\bfseries}{\thechapter}{0.5cm}{\sffamily\Huge}
%\titleformat{\section}[hang]{\sffamily\LARGE}{\thesection}{0.5cm}{}
%\titleformat{\subsection}[hang]{\sffamily\Large}{\thesubsection}{0.5cm}{}
%\titleformat{\subsubsection}[hang]{\sffamily\large}{\thesubsubsection}{0.5cm}{}
%\titleformat{\paragraph}[runin]{\sffamily\normalsize}{}{}{\emph}

%Coloca anexo o apéndice en la Tabla de contenido
\usepackage[toc,page]{appendix}

% Configuración de las páginas en twoside-mode
% Permite ver y configurar los parámetros de la página
\setlength{\voffset}{-0.25in}
\setlength{\headwidth}{467pt}
\setlength{\headheight}{22pt}
\setlength{\oddsidemargin}{0pt}
\setlength{\evensidemargin}{0pt}
\setlength{\marginparwidth}{0pt}
\setlength{\marginparsep}{0pt}
\setlength{\parskip}{2em}
\setlength{\footskip}{20pt}
\setlength{\textheight}{650pt}
\setlength{\textwidth}{467pt}
\setlength{\headsep}{5pt}
\setlength{\parindent}{0pt}
\setlength{\baselineskip}{10pt plus 5pt minus 5pt}
\renewcommand{\theequation}{\thechapter-\arabic{equation}}
\renewcommand{\thefigure}{\textbf{\thechapter-\arabic{figure}}}
\renewcommand{\thetable}{\textbf{\thechapter-\arabic{table}}}


%Define la distancia de la primera linea de un párrafo a la margen
\parindent0cm 

%Espacio entre lineas
\renewcommand{\baselinestretch}{1}

%Para rotar texto, objetos y tablas seite.
\usepackage{rotating}

%Permite incluir mecanismos y reacciones químicas
%\usepackage{tikz}
\usepackage{circuitikz}
\usetikzlibrary{babel}
\usepackage{chemformula}
\usepackage{chemfig}

%\usetikzlibrary{calc,arrows.meta}% per right to e left to
%\tikzset{
%myedge/.style={->, -{Latex[#1]}}
%}

%Fuente de la presentación Ancizar Sans UNAL
%Para usar este compilado en Overleaf se debe usar el compilador XeLaTeX o LuaLaTeX!!
%Menu -> Compiler -> XeLaTeX o LuaLaTeX
%La siguiente línea debe comentarse si desea compilar con pdfLaTeX
%\RequireXeTeX


% Definición de la fuente Ancizar Sans
\newif\ifxetexorluatex

\ifxetexorluatex
  \usepackage{fontspec}
  \usefonttheme{serif}
  \setmainfont{AncizarSans}[Path=./AncizarSans/,Scale=1,Extension=.otf,UprightFont=*-Regular,BoldFont=*-Bold,ItalicFont=*-Italic,BoldItalicFont=*-BoldItalic]
\else
  % Si se compila con pdfLaTeX, cargar la fuente apropiada aquí
  \usepackage[T1]{fontenc}
\fi
% Metadatos del documento
\AtBeginDocument{%
	\hypersetup{
		pdfborder={0 0 0},
		pdfauthor={\studentname},
		pdfsubject={\thesisname}, 
		pdfcreator={\studentname},
		pdfproducer={\studentname},
	}
}

%Carga el símbolo de grado y el de Angstrom
\newcommand{\angstrom}{\textup{\AA}}
\newcommand{\grad}{$^{\circ}$}

%Inicio del documento, no olvide la etiqueta de cierre al final \end{document}
\begin{document}

%Nombres y formatos de títulos, tablas y figuras
%Use \sffamily para dejar con letra Sans Serif, sin etiqueta queda LaTeX clásico
\renewcommand{\listfigurename}{List of Figures}
\renewcommand{\listtablename}{List of Tables}
\renewcommand{\contentsname}{Table of contents}
\renewcommand{\chaptername}{Chapter}
%\renewcommand{\tablename}{\scriptsize \centering \textbf{Tabla}}
%\renewcommand{\figurename}{\scriptsize \centering \textbf{Figura}}
\renewcommand{\appendixname}{Appendix}

%Cambia el nombre de la sección de referencias
\renewcommand{\bibname}{Bibliography}

%Páginas de Presentación del documento - No modificar esto se hace automáticamente
{\newpage
\thispagestyle{empty}
\begin{center}
\begin{figure}
\centering
\epsfig{file=figures/EscudoUN2016,scale=1}%
\end{figure}
\vspace{2.5cm}
\textbf{\Huge \thesisname} \\ 
\vspace{2.5cm}
\textbf{\Large \studentname} \\
\vspace{5.0cm}
\small Universidad Nacional de Colombia \\
\faculty \\
\department \\
\sede, Colombia\\
\issuedate
\newpage 
\thispagestyle{empty}
\vspace{2.5cm}
\textbf{\Huge \thesisname} \\
\vspace{2.5cm}
\textbf{\Large \studentname} \\
\vspace{2.5cm}
\small Tesis presentada como requisito parcial para optar por el título de: \\
{\bfseries \academictitle}\\
\vspace{2.5cm}
Director(a): \\
\director \\
Codirector(a): \\
\codirector \\
\vspace{2.5cm}
Línea de investigación: \\ 
\researchtopic\\
Grupo de investigación: \\
\resgroupone \\
\resgrouptwo \\
\vspace{2.0cm} 
Universidad Nacional de Colombia \\
\faculty \\
\department \\
\issuedate
\end{center}

% Dedicatorias
\newpage
\thispagestyle{empty}
\begin{flushright}
\begin{minipage}{12.5cm}
\noindent
\\[10em]
%Modificar la cita que se quiere agregar
{\Large Cita 01.}
\\[3em]
Autor
\\ \textit{Fuente}
\\[10em]
%Para anular la adición de una segunda cita anule las siguientes lineas desde acá mediante comentario (%)
{\Large \textit{Wenn du es nicht einfach erkl\"{a}ren kannst, hast du es nicht genug verstanden} - Si no eres capaz de explicar algo claramente, es que aún no lo has entendido lo suficiente.}
\\[3em]
Albert Einstein
%Hasta acá!
\end{minipage}
\end{flushright} 

% Declaración de originalidad del texto y del contenido
% No modificar, se hace automáticamente con los comandos ya definidos
\newpage
\chapter*{Declaración}\chaptermark{Declaración}
\par Me permito afirmar que he realizado ésta tesis de manera autónoma y con la única ayuda de los medios permitidos y no diferentes a los mencionados el presente texto. Todos los pasajes que se han tomado de manera textual o figurativa de textos publicados y no publicados, los he reconocido en el presente trabajo. Ninguna parte del presente trabajo se ha empleado en ningún otro tipo de tesis. 
\\[1em]
\sede., \submissiondate
\\[6em]
\rule{6cm}{0.5pt}\\
\studentname
}

%Páginas preámbulo, listado de figuras, tablas y tabla de contenido
{\pagestyle{plain} \pagenumbering{roman}
\setlength{\parskip}{1mm}
\chapter*{Acknowledgments}\chaptermark{Acknowledgments}
\addcontentsline{toc}{chapter}{Acknowledgments}%

This goes to my family, especially to my parents, for providing me with the most sincere and unconditional support I could have ever received. To every friend who stood there when things felt overwhelming and gave me the courage to keep going. Also to every professor and mentor who generously shared their knowledge and advice.
\\

A special thanks goes to Professor Fernando Cristancho for introducing me to an extremely nurturing working environment, filled with people without whom this could never have come to be. I am also deeply grateful to Professor Spencer N. Axani, for opening the doors to not only his group but also to immensely rewarding experiences and knowledge I will carry with me forever.
\\

Last but not least I want to thank me for believing in me, I want to thank me for doing all this hard work, for never quitting and for being me at all times.   
\chapter*{Listado de símbolos y abreviaturas}\chaptermark{Listado de símbolos y abreviaturas}
\addcontentsline{toc}{chapter}{Listado de símbolos y abreviaturas}
\newpage
\chapter*{Resumen}\chaptermark{Resumen}
\addcontentsline{toc}{chapter}{Resumen}%
\textbf{\Huge \ \strut CosmicWatch: Los Detectores de Muones de Escritorio, explorando la espectroscopia gamma \strut} %strut keeps interline spacing consistent
\vspace{.5cm}
\par El presente trabajo se concentra en el mejoramiento de CosmicWatch: Los Detectores de Escritorio de Muones, con el objetivo de obtener una resolución en energías suficiente para realizar espectroscopia gamma. Dadas ciertas limitaciones actuales en la electrónica, como tiempo muerto y baja rata de conteo, junto con la baja resolución de los centelladores plásticos usados hasta ahora, este trabajo explora entonces el uso de una RaspberryPi Pico y un cristal centellador basado en Lutecio y dopado con Cerio (LYSO:Ce), logrando resolución en energías de $13.5\%$ para 511 \unit{\kilo\eV} al medir con un osciloscopio Rohde\&Schwarz RTO6, $28.3\%$ con módulos NIM, y 36.5\% al procesar la señal con nuestros diseños de amplificación y detección de picos.
\\[2cm]
\textbf{Palabras clave:} \palabrasclave

\newpage 
\chapter*{Abstract}\chaptermark{Abstract}
\addcontentsline{toc}{chapter}{Abstract}%
\textbf{\Huge \ \strut CosmicWatch: The Desktop Muon Detectors, exploring gamma-ray spectroscopy \strut} %strut keeps interline spacing consistent
\vspace{.5cm}
\par The present work focuses on the improvement of CosmicWatch: The Desktop Muon Detectors, with the goal of achieving sufficient energy resolution to allow for gamma spectroscopy. Due to current limitations in the electronics, such as dead time and small sample rate, in addition to the low energy resolution of previously used plastic scintillators, this work explores the use of a RaspberryPi Pico together with a Cerium-doped Lutetium-based scintillation crystal (LYSO:Ce), achieving energy resolutions of $13.5\%$ at 511 \unit{\kilo\eV} while sampling data with a Rohde\&Schwarz RTO6 oscilloscope, $28.3\%$ with NIM modules, and 36.5\% while using our amplifier and peak-detector designs.
\\[2cm]
\textbf{Keywords:} \keywords

%\newpage 
%\chapter*{\sffamily Zusammenfassung}
%\addcontentsline{toc}{chapter}{Zusammenfassung}%
%\par Zusammenfassung texte.
%\par 
%\\[2cm]
%\textbf{Schlüsselwörter:} \schlusselworter
\tableofcontents 
%\addcontentsline{toc}{chapter}{Table of contents}
\listoffigures 
%\addcontentsline{toc}{chapter}{List of Figures}
\listoftables 
%\addcontentsline{toc}{chapter}{List of Tables}
\clearpage
}

{\pagenumbering{arabic}
\setlength{\parskip}{\baselineskip}
%Incluir secciones del documento de aquí en adelante
%Use \include para incluir desde una página nueva e \input para incluir sin salto de página
\include{chapters/introduction}
\chapter{Physical aspects}

Atomic nuclei have multiple energy states, many of which are not stable, forcing them to release some of their energy in order to reach a stable state. One decay however, does not always lead to a direct transition into a stable state, it may even require the atom to transform into another depending on what way it has released its energy by. The theory on this chapter regarding energy and particle emissions from decaying nuclei is based on the detailed information shown in the first chapters of ``Radiation Detection and Measurement'' by Glenn F. Knoll \cite{knoll2010radiation}. On the other hand, radiation is not only produced when atoms decay, it is constantly raining down upon us from the cosmos, this kind of radiation can also be detected with a CosmicWatch, making it also necessary to explore. This chapter aims to provide an overview of some of the mechanisms through which nuclei reach stable states and what cosmic radiation is. Chapter \ref{chap:detection_methods} provides an overview of how this can be used to take interesting measurements with CosmicWatch.

\section{Radioactivity}

It is first necessary to understand the concept of activity. Not all atoms take the same time to decay, each atom has its own constant $\Gamma$ which determines how likely it is to decay per unit of time. If one has an initial total of $N_0$ atoms, after a while it will be reduced due to the constant decay of atoms in the sample, this rate of change is given by the universal law of radioactive decay.
\begin{equation}
    \frac{dN(t)}{dt} = -\Gamma N(t)
\end{equation}

From this, it is easy to find that the number of remaining unstable atoms follows an exponential law
\begin{equation}
    N(t) = N_0 e^{-\Gamma t}
\end{equation}

The activity $A(t)$ of a radioactive source is given by how many decays occur per unit of time. This can be therefore obtained by multiplying the number of atoms $N(t)$ by the probability of decay per unit of time $\Gamma$.
\begin{equation}
    A(t) = \Gamma N_0 e^{-\Gamma t}
\end{equation}

The most common units for activity are the \textit{curie} (Ci) and the \textit{becquerel} (Bq), a becquerel represents one disintegration per second, while a curie represents $3.7\times10^{10}$ disintegrations per second ($\approx$ the activity of one gram of $^{226}Ra$). Under these definitions, the conversion between these units is given by the following relation:
\begin{eqnarray}
    1 \text{~Bq} = 2.703\times10^{-11} \text{~Ci}
\end{eqnarray}

The time constant $\Gamma$ is often expressed in terms of the atom's lifetime $\tau$ under the relation $\Gamma=1/\tau$. This means that $\tau$ is the time it takes to reduce a sample of $N_0$ by a factor of $1/e$, clearly $N(\tau)=N_0e^{-\Gamma \tau} = N_0/e$. On the other hand there also exists a constant called half-life $T_{1/2}$, which represents the time it takes to reduce the sample by half, Sodium 22 for example has a half-life of 2.605 years. They can both be related by doing $T_{1/2} = \ln (2)\tau$.

\subsection{Gamma emission}

\begin{figure}
  \centering
  \begin{subfigure}[t]{0.45\textwidth}
    \includegraphics[width=\textwidth]{physical_aspects/22Na-decay.pdf}
    \caption{\label{sfig:22Na}}
  \end{subfigure}
  \begin{subfigure}[t]{0.425\textwidth}
    \includegraphics[width=\textwidth]{physical_aspects/57Co-decay.pdf}
    \caption{\label{sfig:57Co}}
  \end{subfigure}
  \medskip
  \centering
  \begin{subfigure}[t]{0.425\textwidth}
    \includegraphics[width=\textwidth]{physical_aspects/60Co-decay.pdf}
    \caption{\label{sfig:60Co}}
  \end{subfigure}
  \begin{subfigure}[t]{0.425\textwidth}
    \includegraphics[width=\textwidth]{physical_aspects/137Cs-decay.pdf}
    \caption{\label{sfig:137Cs}}
  \end{subfigure}
  \caption{\label{fig:decay_schemes}Decay schemes for some isotopes used while testing the CosmicWatch. Only the main decay channels are included for clarity and simplicity. Energies [\unit{\kilo\eV}] for every level are shown in black. Branching ratios and decay mechanisms are shown in \textcolor{blue}{blue}. Gamma decays are represented with a \textcolor{red}{red} arrow also with its corresponding branching ratio.}
\end{figure}

Unstable nuclei have multiple channels to release their energy through, the conditions that determine what channels an atom can use are not studied here, but rather the subsequent effects of such channels. An atom can decay by emitting gamma rays, alpha particles, neutrons, or protons, it can also undergo beta $\beta^{\pm}$ decay, Internal Conversion, and Electron Capture among others. This work will focus on beta decay and Electron Capture since these are the preferred channels of decay of the radioactive sources used to test CosmicWatch.

\subsubsection{Beta decay}

There are two types of beta decay, they are represented by the following reaction schemes:
\begin{align}
  \beta^+ &:=~ ^A_ZX \rightarrow ~ ^A_{Z-1}Y + e^+ + \nu \\
  \beta^- &:=~ ^A_ZX \rightarrow ~ ^A_{Z+1}Y + e^- + \bar{\nu}
\end{align}

Where the symbols follow the nuclear notation, $X$ and $Y$ represent the initial and final elements, $A$ is the atomic number, $Z$ the nuclear charge, $e^{\pm}$ are a positron or electron, and $\nu/\bar{\nu}$ are a neutrino/antineutrino, Fig. \ref{fig:decay_schemes} shows some examples of these processes. Note for instance the case of $^{22}_{11}$Na Fig. \ref{sfig:22Na}, it undergoes $\beta^+$ $90\%$ of the time it decays, by the nuclear notation one can tell that the initial and final elements are Sodium and Neon respectively. In this process, a proton turns into a neutron, which is why the product element has $Z=11-1=10$ while maintaining $A=22$. It is important to also note that the total charge has to be conserved after the reaction occurs, which is why a positron $e^+$ is produced.

Alongside the positron/electron, a neutrino/antineutrino is ejected from the nucleus which, due to its extremely small interaction probability with matter, can not be detected. However, the negligible neutrino/antineutrino-matter interactions do not mean that their presence in the reaction does not have effects. The energy of the system also has to be conserved, since the particles $e^{\pm}$ and $\nu/\bar{\nu}$ are all ejected from the nucleus, higher or smaller portions of the total energy can be taken by the neutrino/antineutrino, which leaves multiple possible energy values for the positron/electron, resulting in continuous energy spectra.

\begin{figure}[H]
    \centering
    \begin{subfigure}[t]{0.45\textwidth}
      \includegraphics[width=\textwidth]{physical_aspects/22Na-beta-spectrum.jpg}
      \caption{\label{sfig:22Na_beta_spectra}}
    \end{subfigure}
    \begin{subfigure}[t]{0.45\textwidth}
      \includegraphics[width=\textwidth]{physical_aspects/60Co-beta-spectrum.jpg}
      \caption{\label{sfig:60Co_beta_spectra}}
    \end{subfigure}
    \caption{\label{fig:beta_spectra}positron/electron energy spectra for  \subref{sfig:22Na_beta_spectra} $^{22}$Na $(\beta^+)$ and  \subref{sfig:60Co_beta_spectra} $^{60}$Co $(\beta^-)$. Taken from \cite{IAEA}.}
\end{figure}

\subsubsection{Electron Capture}

The process of Electron Capture is analogous to $\beta^+$ decay, here an electron from de $K$ shell (or $L$, $M$, \dots) is captured by a proton in the nucleous, which then turns into a neutron while emitting a neutrino with a characteristic energy. The effect of this decay in the nucleous is the same as in $\beta^+$: (Z, A)$\rightarrow$(Z-1, A). The decay scheme is represented below:

\begin{equation}
  \text{EC} :=~ p + e^- \rightarrow ~ n + \nu
\end{equation}

Since this process leaves a vacancy in the lower shells of the electron cloud, an electron in an upper shell can decay to fill the space, resulting in the emission of characteristic X-rays.

\subsection{Light-matter interactions}

This section shows a review of the most important processes that govern gamma-ray interactions with matter, them being photoelectric effect, Compton scattering, and pair production.

\subsubsection{photoelectric effect}

This effect was first described by Alber Einstein in \cite{einstein1905heuristic}, is is the emission of electrons from an atom due to the absortion of light. The electrons are emitted with an energy close to that of the absorved photon $E_\gamma$ and is given by equation \eqref{eq:photoelectric}
\begin{equation}
  E_{e^-} = E_\gamma - E_b \label{eq:photoelectric}
\end{equation}
Where $E_b$ is the binding energy of the electron in the atom. $E_b$ will therefore depend on the electrons original shell, since outter shells have lower binding energies.

\subsubsection{Compton scattering}

\begin{figure}[H]
  \centering
  \includegraphics[width=.4\textwidth]{physical_aspects/Compton-scattering.pdf}
  \caption{\label{fig:Compton_scattering_diagram}Gamma-ray Compton scattering diagram.}
\end{figure}

Compton scattering osccurs most often when a photon interacts with an electron in the sensitive material, it can be understood as a colision between the two, where the electron is considered initialy static and then gains part of the photons energy $E_0$ after deflecting it. Fig \ref{fig:Compton_scattering_diagram} shows an example of a gamma-ray being scattered by an electron, the recoil electron is ejected with an angle $\phi$ while the scattered gamma ray travels with an angle $\theta$. It can be shown that the energy of the scattered photon $E_{1}$ and the recoil electron $E_e$ follow equations \eqref{eq:compton}
\begin{align}
  E_{1} &= \frac{E_0}{1+\epsilon(1-\cos\theta)} \label{eq:compton},~ & E_e &= E_0 - E_1 = E_0\frac{\epsilon(1-\cos\theta)}{1+\epsilon(1-\cos\theta)};~ & \epsilon&=\frac{E_0}{m_{e}c^2} 
\end{align}
where $m_e c^2$ is the resting energy of the electron (511 \unit{\kilo\eV}). It is easy to see that high scattering angles greatly reduce the photon's energy, meaning that $E_e$ will be higher. For $\theta=\pi$ and $\phi=0$, it is therefore clear that the electron will gain the maximum energy possible, which is lower than $E_0$, this will be inportant while detecting gammas, since the photomultipliers collect the energy carried by electrons, which is further explored in Chapter \ref{chap:detection_methods}.

\subsubsection{Pair production}

This effect is the creation of a particle and its antiparticle from a neutral bosson, most often refer to when an electron and a positron are produced by a photon. In order for this to occur, the initial photon needs to have an energy higher to that of the resting pair ($2m_e c^2=1022$ \unit{\kilo\eV}), any exceding energy will turn into kinetic energy for the pair. 

\subsubsection{Lineal attenuation coeficient}

\section{Cosmic Radiation}

\section{Particle interactions with matter}
\chapter{Detector description}

%versions of the detector
\section{History}

\section{Plastic vs. LYSO}

\section{Power Consumption}

\section{KiCad}

\section{Accessories}

\section{3D printed case}

In order to hold the crystal in place on the SiPM PCB, it was necessary to design a 3D printed case -see Fig. \ref{fig:3d_case_desing} for an image of the 3D desing on inventor. With this we made sure that the crystal would not move with respect to the SiPM, preventing scratches and providing a more stable optical coupling with the photomultiplier.

\begin{figure}[H]
    \centering
    \includegraphics[width=0.6\textwidth]{Detector_description/3d_case_inventor.png}
    \caption{3D model of the LYSO case made on inventor, the .cad files can be found on the repository \href{https://github.com/anvargasl/CosmicWatch-gamma-spectroscopy-PCB}{CosmicWatch-gamma-spectroscopy-PCB}.}
    \label{fig:3d_case_desing}
\end{figure}

The design keeps in mind that the crystal has to be wrapped in teflon tape to increase reflectivity, which is why it comes in two pieces that come together around the crystal, lowering the risk of tears. Once the crystal is placed in the case it can be kept together by means of electrical tape.

Before using teflon tape, the crystal was wrapped in tin foil, which made tears common (Fig. \ref{fig:tin_foil_tear}) and greatly impacted the quality of the spectra that could be obtained with the detector. 

\begin{figure}[H]
    \centering
    \begin{subfigure}[t]{0.45\textwidth}
      \includegraphics[width=\textwidth]{Detector_description/LYSO-wrapped.jpeg}
    \end{subfigure}
    \begin{subfigure}[t]{0.45\textwidth}
      \includegraphics[width=\textwidth]{Detector_description/tin_foil_tear.jpg}
    \end{subfigure}
    \caption{\label{fig:tin_foil_tear}Tin foil tear.}
\end{figure}

Teflon tape seems to solve the tearing problem. However, with the intention to reduce the risk of tearing the teflon, multiple iterations of the case were desined (Fig. \ref{fig:3d_previous_desings})

\begin{figure}[H]
    \centering
    \includegraphics[width=0.6\textwidth]{Detector_description/Holder-designs.jpeg}
    \caption{3D printed cases tested to reduce risk of teflon tearing.}
    \label{fig:3d_previous_desings}
\end{figure}


\chapter{Detection methods} \label{chap:detection_methods}

%versions of the detector
\section{Scintillation}

\section{Single photon detectors}

\section{PMTs}

Photomultipliers are very useful tools in particle detection. Since some detection methods rely on the measurement of photons it is very important to make use of every single one. Some PMs are capable of creating measurable electrical signals from single photon interactions, provoking an electron avalanche that can be easily read with various devices. Photomultiplier Tubes are the most common light amplifiers paired with scintillators, they are however bulky and require very high voltages to operate. An in-depth description of this and other types of Photomultipliers can be found in Ref. \cite{knoll2010radiation} Chap. 9.

\section{SiPM advantages}

CosmicWatch is meant to be a portable, easy-to-build, and economical device, making PMTs not suitable for the overall intention of the project. This, however, is fully solved by Silicon Photomultipliers, their small form factor, relatively low operating voltage, and good efficiency at wavelengths near the emission maxima of common scintillating materials, turn them into an ideal candidate for use in this project. For further details on SiPMs also read Ref. \cite{knoll2010radiation} Chap. 9 and Ref. \cite{Onsemi_SiPM_intro}.

A Silicon Photomultiplier is an array of Geiger-mode avalanche-photodiodes. To make this more digestible we can introduce these concepts one by one. A photodiode is essentially made of a P-N semiconductor junction, where scintillation photons can create new electron-hole pairs that will be accelerated by the bias voltage applied to the diode. However, this process alone produces small signals, making a preamplification stage necessary. Avalanche photodiodes alleviate this problem by increasing the bias voltage, therefore giving the photoelectrons higher energies, allowing them to collide and produce new electron-hole pairs that will also be detected. When the bias voltage rises sufficiently high, the regions where photoelectrons multiply merge together, making one big avalanche, this is the so-called Geiger mode of APDs, and this high voltage receives the name of Breakdown Voltage. Geiger mode APDs can produce a large output from a single scintillation photon. The difference Between Bias and Breakdown Voltage is known as Over-voltage.

SiPMs however, also have some disadvantages, since they are based on the transport of photoelectrons, the output can vary with temperature, as high temperatures can create electron-hole pairs that will be read as a photon interaction. These events are called dark pulses and represent a random noise in the signal.

The SiPM used and tested throughout this work is a MicroFJ-300XX-TSV by Onsemi \cite{Onsemi_SiPM} (Fig. \ref{fig:Onsemi_SiPM}), which has an active area of $3.07 \times 3.07$ \unit{\mm\squared} and a microcell size of 35 \unit{\micro\m}. In this case, the operating voltage ranges between 25.2 and 30.7 \unit{\V}, which can be obtained using a DC to DC booster (see Section \ref{sec:DC_DC}), no longer requiring the high voltages of PMTs. We are currently operating at an Over-voltage of 4 V, which according to Ref. \cite{Onsemi_SiPM} produces a relatively high dark count rate (50-150 \unit{\kilo\Hz\per\mm\squared} at 21 $^\circ$C), potentially affecting energy resolution.

\begin{figure}[H]
    \centering
    \includegraphics[width=0.8\textwidth]{detection_methods/MicroFJ−300XX−TSV.jpeg}
    \caption{Onsemi MicroFJ-300XX-TSV. Taken from \cite{Onsemi_SiPM}.}
    \label{fig:Onsemi_SiPM}
\end{figure}
\include{chapters/electronics}
\chapter{Geant4 Simulation}

In order to provide a new interactive approach to CosmicWatches for new users a scintillation simulation has been devised using the Geant4 toolkit. With this, we intend to showcase some of the physical aspects that may be studied with the detector. This project is still in development and can be found on the GitHub repository \href{https://github.com/spenceraxani/CosmicWatch_Geant4}{CosmicWatch\_Geant4}. This section aims to give a brief explanation of the inner workings of the simulation as well as the Geant4 toolkit.

\section{What is Geant4?}

As defined by the authors \cite{Geant4}, Geant4 is a simulation toolkit based in \texttt{C++}, to model the passage of particles through matter, allowing the creation of custom detector geometries, tracking, and hit recollection, it includes multiple physics models for processes covering a wide range of specialties, such as electromagnetic, hadronic, and optical processes. Its main areas of application are high energy, nuclear, and accelerator physics, as well as medical and space science.

A comprehensive \href{https://geant4-userdoc.web.cern.ch/UsersGuides/InstallationGuide/html/}{installation guide} has been made by the authors, the toolkit is available on Windows, MacOS, and Linux. We recommend building and installing it from source using CMake, since it provides great control over the specific configurations available during installation.

\section{Geometry}

As noted before, Geant4 allows the creation of custom geometries, which in our use case is great, since it provides a platform suitable for testing multiple CosmicWatch configurations as shown in Sections \labelcref{sec:collected_produced,sec:SiPM_placement,sec:Scint_size,sec:cos_squared}.

In general, Geant4 requires a mother volume in which the simulation will take place, in this case, we have chosen a $1\times1\times1$ \unit{\m\cubed} box of \texttt{G4\_AIR}\footnote{This and all other materials used in the simulation, are built from the \textit{Geant4 Material Database}, accessed through the \texttt{G4NistManager} class}. For practical effects, the SiPM is also modeled as a \texttt{G4\_AIR} box of $3\times3\times1$ \unit{\mm\cubed}. The scintillating material is a $5\times5\times1$ \unit{\cm\cubed} box. The sizes of all these elements can of course be changed (as explored in Section \labelcref{sec:Scint_size}) the settings detailed above are just the default configuration in \texttt{headers/construction.hh}. Fig. TALES shows the SiPM and scintillator placement inside the mother volume.

%----------------include detector geometry screenshot

As discussed in \labelcref{sec:self_radiation}, LYSO crystals contain $^{176}$Lu, a naturally occurring radioactive isotope emitting betas. LYSO crystals can absorb their own radiation, resulting in a constant background, this feature however, has not yet been included in the simulation, which is why the scintillating material used is the same as in previous versions of CosmicWatch, a plastic scintillator based on Polyvinyltoluene, the material name in the \textit{Geant4 Material Database} is \texttt{G4\_PLASTIC\_SC\_VINYLTOLUENE}.

\section{Muons going through the scintillator}

\section{Simulation results}

\subsection{Photons collected vs. produced}\label{sec:collected_produced}

\subsection{Optimum SiPM placement}\label{sec:SiPM_placement}

\subsection{Scintillator sizes}\label{sec:Scint_size}

\subsection{Cosine squared law}\label{sec:cos_squared}

\subsection{Simulated Spectra}
\chapter{Measurements}\label{chap:measurements}

\section{Rohde\&Schwarz RTO6 oscilloscope}

\section{CosmicWatch electronics}

\section{NIM}
\chapter{Ongoing work and future directions}

\section{Odd features in Cesium spectra}

\section{Adding LYSO radioactivity to Geant4}
\chapter{Conclusions}

\begin{itemize}
    \item CosmicWatch: The Desktop Muon Detectors show promising potential in gamma spectroscopy. The integration of Silicon photomultipliers with LYSO crystals represents a significant improvement from previous versions that used plastic scintillators.\item Despite technical challenges, it was proven that the detector is capable of resolving peaks, this progress was achieved through meticulous optimizations in hardware and software, marking a substantial advancement towards a fully functional gamma spectrometer, achieving an energy resolution of 36.5\% at 511 keV.
    \item The multithreading approach, made possible with the use of a Raspberry Pi Pico, enabled the detector to keep up with the high activities of our radioactive sources, significantly decreasing dead time and therefore boosting its sample rate.
    \item It was shown that our electronics can be coupled with other acquisition systems, such as the Rohde\&Schwarz RTO6 oscilloscope and NIM modules, providing flexibility and opening the door for accessible and user-friendly detectors.
    \item Further research needs to be done in order to take full advantage of the ADC and improve linearity in the electronics. The implementation of the \texttt{C/C++} SDK is a promising first approach.
\end{itemize}

The enhancements to CosmicWatch were obtained while remaining affordable and user-friendly, aligning with our primary goal of keeping science available to students, educators, researchers, and enthusiasts alike, inviting them to delve into the world of particle and nuclear physics, electronics, programming, and data analysis. Furthermore, we outline plausible paths to build upon our work and encourage CosmicWatch users to use these materials as a stepping stone to further contribute to the project.

%Inicio del apéndice o anexos
\begin{appendix}
\chapter{RaspberryPi Pico code}\label{app:RP_Pico_code}

All code included here can be found in the repository \\ \href{https://github.com/anvargasl/CosmicWatch-gamma-spectroscopy-RP}{https://github.com/anvargasl/CosmicWatch-gamma-spectroscopy-RP}.

\section{ADC Calibration}

In Chapter \ref{chap:Electronics} we explored the amplification and peak detection stages along with some of their shortcomings, in order to attenuate the effects of the peak-detector's nonlinearity a calibration of the ADC channels must be done, this allows to correlate the incoming SiPM pulse amplitude with a specific ADC reading.

In order to calibrate the detector response one can use an Arbitrary Function Generator, the easiest way to recreate a SiPM pulse with one of these tools is to save it with an oscilloscope and then upload it to the generator. If this is not possible, one can use a square pulse shape and adjust the rise and fall times in order to produce a signal that resembles a SiPM pulse (make sure that the amplifier responds similarly for the simulated signal). The artificial pulse can be fed to the detector through the SMA connector (make sure to disconnect the PM before doing this).

The code shown bellow can be used to calibrate the detector. The list \texttt{Voltages} contains preliminary pulse amplitudes to be used for calibration, five samples are taken for every pulse in the \texttt{read\_ADC} interrupt routine, this is done for \texttt{buffer\_size=100} pulses before continuing with the next amplitude (a five-second window is given to the user to change the amplitude before continuing). Make sure that the pulse frequency is not too high, this might cause pulses to fall on top of each other and therefore producing a poor calibration.

\lstinputlisting[language=python]{./appendixes/code/calibration.py}

\section{Ring Buffer}

The main goal of the code is to take full advantage of both cores in the Pico, in order to do this a ring buffer class has been implemented in order to make sure that core 0 will not overwrite unsaved data that is yet to be read by core 1. This module is adapted to our use case from the original code by Peter Hinch under MIT license found at \href{https://github.com/peterhinch/micropython-async/blob/master/v3/primitives/ringbuf_queue.py}{micropython-async}.

\lstinputlisting[language=python]{./appendixes/code/RingbufQueue.py}

\section{OLED module}

In order to facilitate the use of the SSD1306 OLED display a small module for text display and erase has been included. It also allows to print a pixel art of the CosmicWatch logo as a bootscreen. Please note that some lines in the file have been omitted since they contain some very long bytearrays.

\lstinputlisting[language=python,linerange={1-32,34-104}]{./appendixes/code/OLED.py}%
\end{appendix}

%Permite visualizar la bibliografía en la tabla de contenido
%\addcontentsline{toc}{chapter}{References} %Literatura Citada

%\let\OLDthebibliography=\thebibliography
%\def\thebibliography#1{\OLDthebibliography{#1}}
%{\scriptsize
%\pagestyle{plain}
% Nombre del documento donde se almacenan las referencias
%\bibliography{Referencias}
%\nocite{*}
%\cleardoublepage
%}
%}

\bibliographystyle{unsrt}

\bibliography{bibliography}

\end{document}